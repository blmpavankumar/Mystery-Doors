\chapter{IMPLEMENTATION}

This chapter gives a brief description about implementation details of the
system by describing each module with its code skeleton. The language used for implementation is C\# and Platform used is Windows phone 7.

\section{Prototype : 1}
\hspace{1cm}In prototype 1 implementation of introduction,  level 1 and level 2 was been carried.
\subsection{Introduction}
\ttfamily \hspace{1cm}
Code skeleton:\\
\\

namespace MD\\
\{\\
    
    public partial class Page2 : PhoneApplicationPage\\
   {\\
        public Page2()\\
        {\\
            InitializeComponent()\\;

            
        }\\
        void leve1()\\
        {\\

        }\\
        public void doStuff()\\
        {\\
           
            System.Threading.Thread startupThread =\\
                             new System.Threading.Thread(new System.Threading.ThreadStart(pDelay2));\\
            startupThread.Start();\\
        }\\
        public void doStuff2()\\
        {\\

            System.Threading.Thread startupThread =\\
                             new System.Threading.Thread(new System.Threading.ThreadStart(pDelay3));\\
            startupThread.Start();\\
        }\\
        public void doStuff3()\\
       {\\
            button2.Visibility = Visibility;\\
        }\\
        BitmapImage im2 = new BitmapImage(new Uri("/MD;component/Images/sds.png", UriKind.Relative));\\
        BitmapImage im1 = new BitmapImage(new Uri("/MD;component/Images/z.png", UriKind.Relative));\\
        BitmapImage im3 = new BitmapImage(new Uri("/MD;component/q.png", UriKind.Relative));\\
        BitmapImage im4 = new BitmapImage(new Uri("/MD;component/w.png", UriKind.Relative));\\
        BitmapImage im5 = new BitmapImage(new Uri("/MD;component/Xtvv.jpg", UriKind.Relative));\\
        BitmapImage im6 = new BitmapImage(new Uri("/MD;component/dhaka2.png", UriKind.Relative));\\
        BitmapImage im7 = new BitmapImage(new Uri("/MD;component/ww.png", UriKind.Relative));\\
        BitmapImage im8 = new BitmapImage(new Uri("/MD;component/v1.png", UriKind.Relative));\\
        BitmapImage im9 = new BitmapImage(new Uri("/MD;component/v2.png", UriKind.Relative));\\
 
        private void button1_Click_1(object sender, RoutedEventArgs e)\\
        {\\
            
            System.Threading.Thread startupThread =\\
                              new System.Threading.Thread(new System.Threading.ThreadStart(pDelay));\\
            startupThread.Start();\\
           // DoStuff();\\
        }\\
        private void button1_Click(object sender, RoutedEventArgs e)\\
        {\\
            
           
    
        }\\
        void pDelay3()\\
        {\\
            int k = 20, j = 1, m = 0;\\
            Boolean flag2 = true;\\
            for (int i = 0; i < 50; i++)\\
            \{\\
                System.Threading.Thread.Sleep(150);\\

                this.Dispatcher.BeginInvoke(() =>\\
               \{\\
                    image1.Source = im5;\\

                    \{\\
                        image3.Visibility = Visibility;\\
                        if (flag2)\\
                        \{\\
                            if (image3.Height != 0 || image3.Width != 0)\\
                            \{\\
                                image3.Height = image3.Height - 10;\\
                                image3.Width = image3.Width - 10;\\
                            \}\\
                            else\\
                           \{\\
                                button2.Visibility = Visibility;\\
                            \}\\
                            
                        \}

                    \}
                \}\\);\\
            \}\\
        //    doStuff3();\\
        \}\\
        void pDelay2()\\
        \{\\
            int k=20,j=1,m=0;\\
             Boolean flag2 = true;\\
            for (int i = 0; i < 50; i++)\\
            \{\\
                System.Threading.Thread.Sleep(150);\\

                this.Dispatcher.BeginInvoke(() =>\\
                \{\\
                    image1.Source = im5;\\
                    image2.Source = im3;\\
                   
                    \{\\
                        image3.Visibility = Visibility;\\
                        if (flag2)\\
                        \{\\
                            if (image3.Height < 110 || image3.Width < 130)\\
                            \{\\
                                image3.Height = image3.Height + 10;\\
                                image3.Width = image3.Width + 10;\\
                            \}\\
                            else\\
                            \{\\
                                //NavigationService.Navigate(new Uri("/Page2a.xaml", UriKind.Relative));\\
                                if (j == 0)\\
                                \{\\
                                    image3.Source = im8;\\
                                    j=1;\\
                                \}\\
                                else\\
                                \{\\
                                    image3.Source = im9;\\
                                    j = 0;\\
                                \}\\
                            \}
                        \}
                           
                        \}\\                     
                \}\\);\\
            \}\\
            doStuff2();\\
        \}\\

        void pDelay()\\
        \{\\
            int j = 0, k = 6,m=550,n=550;\\
            Boolean flag = true;\\
           
            for (int i = 0; i < 50; i++)\\
            \{\\

                System.Threading.Thread.Sleep(150);\\

                this.Dispatcher.BeginInvoke(() =>\\
                \{\\
                    if(flag)\\
                    \{\\

                        if (j == 0)\\
                        \{\\
                            image2.Source = im1;\\
                            image2.Margin = new Thickness(k, 26, 0, 0); k = k + 10;\\
                            j = 1; m=m-10;\\
                            if (m == 0)\\
                            \{\\
                                flag = false;\\
                            \}\\


                        \}\\
                        else\\
                        \{\\
                            image2.Source = im2;\\
                            image2.Margin = new Thickness(k, 26, 0, 0); k = k + 10;\\
                            j = 0; m=m-10;\\
                            if (m == 0)\\
                            \{\\
                                flag = false;\\
                            \}\\
                        \}
                    \}


                    else\\
                    \{\\
                        if (j == 0)\\
                        \{\\
                            image2.Source = im3;\\
                            image2.Margin = new Thickness(k, 26, 0, 0); k = k - 10;\\
                            j = 1; \\
                            
                                m = m + 10;\\
                                if (m == 550)\\
                                \{\\
                                    flag = true;\\
                                \}\\
                        \}\\
                        else\\
                        \{\\
                            image2.Source = im4;\\
                            image2.Margin = new Thickness(k, 26, 0, 0); k = k - 10;\\
                            j = 0; m = m + 10;\\
                            if (m == 550)\\
                            \{\\
                                flag = true;\\
                            \}\\
                        \}
                    \}
                   



                \}\\);\\
            \}\\
           doStuff();\\
        \}\\
        
        void azzDelay1()\\
        \{\\

            for (int i = 0; i < 400; i++)\\
            \{\\
                Thread.Sleep(250);\\
                

                this.Dispatcher.BeginInvoke(() =>\\
                \{\\
                                            

                \}\\);\\
            \}\\
            
        \}\\

        private void image2_ImageFailed(object sender, ExceptionRoutedEventArgs e)\\
        \{\\

        \}\\

        private void button2_Click(object sender, RoutedEventArgs e)\\
        \{\\
            NavigationService.Navigate(new Uri("/Page3.xaml", UriKind.Relative));\\
        \}\\

        
    \}
\}

\rmfamily



\subsection{Level1}\\
\ttfamily \hspace{1cm}\\


namespace MD\\
\{\\
    public partial class Page4 : PhoneApplicationPage\\
    \{\\
        public Page4()\\
        \{\\
            InitializeComponent();\\
        \}\\
        private void image1_ImageFailed(object sender, ExceptionRoutedEventArgs e)\\
        \{\\

        \}\\
        int i = 3, m = 2;\\
        private void button3_Click(object sender, RoutedEventArgs e)\\
        \{\\
            string s1;\\
            s1 = textBox1.Text;\\
            if (s1 == "cross breed")\\
            \{\\
                textBlock2.Visibility = Visibility.Collapsed;\\
                textBlock3.Visibility = Visibility.Collapsed;\\
                button1.Visibility = Visibility;\\
                textBlock5.Text = "You got it Right.Press Continue";\\
                
           \}\\
            else\\
            \{\\
                textBlock5.Text = "Sorry You are wrong !!!";\\
                textBox1.Text = "";\\
                if (m != 0)\\
                \{\\

                    if (m == 2)\\
                    \{\\
                        
                        textBlock5.Text = "Sorry You are wrong !!!";\\
                        textBlock3.Text = "2"; m--;\\
                    \}\\
                    else if (m == 1)\\
                    \{\\
                        textBlock5.Text = "Sorry You are wrong again !!!";\\
                        textBlock3.Text = "1"; m--;\\
                    \}\\
                \}
                else
                \{\\
                    NavigationService.Navigate(new Uri("/Page35.xaml", UriKind.Relative));\\
                \}\\
            \}\\
        \}\\
        private void button1_Click(object sender, RoutedEventArgs e)\\
        \{\\
            NavigationService.Navigate(new Uri("/Page5.xaml", UriKind.Relative));\\
        \}\\
    \}
\}




\rmfamily



\subsection{Level2}
\ttfamily \hspace{1cm}
\rmfamily
Code skeleton:\\
namespace MD\\
\{\\
    public partial class Page10 : PhoneApplicationPage\\
    \{\\
        public Page10()\\
        \{\\
            InitializeComponent();\\
        \}\\
        int i = 3, m = 2;\\
        private void image1_ImageFailed(object sender, ExceptionRoutedEventArgs e)\\
        \{\\
            
        \}\\
        private void button1_Click(object sender, RoutedEventArgs e)\\
        \{\\
            NavigationService.Navigate(new Uri("/Page11.xaml", UriKind.Relative));\\
        \}\\

        private void button3_Click(object sender, RoutedEventArgs e)\\
        \{\\
            string s1;\\
            s1 = textBox1.Text;\\
            if (s1 == "top secret")\\
            \{\\
                button1.Visibility = Visibility;\\
                textBlock5.Text = "You got it Right.Press Continue";\\
            \}\\
            else\\
            \{\\                textBlock5.Text = "Sorry You are wrong !!!";\\
                textBox1.Text = "";\\
                if (m != 0)\\
                \{\\

                    if (m == 2)\\
                    \{\\
                        textBlock5.Text = "Sorry You are wrong !!!";\\
                        textBlock3.Text = "2"; m--;\\
                    \}\\
                    else if (m == 1)\\
                    \{\\
                        textBlock5.Text = "Sorry You are wrong again !!!";\\
                        textBlock3.Text = "1"; m--;\\
                    \}\\
                \}
                else\\
                \{\\
                    NavigationService.Navigate(new Uri("/Page35.xaml", UriKind.Relative));\\
                \}
            \}
        \}
    \}
\}\\





\section{Prototype : 2}
\hspace{1cm}In prototype  implementation of level3, level 4 and level 5  was been carried.

\subsection{Level3}
\ttfamily \hspace{1cm}Describe the component and write the actual code for the specific component.
Code skeleton:\\
\\
namespace MD\\
\{
    public partial class Page16 : PhoneApplicationPage\\
    \{
        public Page16()\\
        \{
            InitializeComponent();\\
            System.Threading.Thread startupThread =\\
                             new System.Threading.Thread(new System.Threading.ThreadStart(timeDec));\\
            startupThread.Start();\\
        \}
        Boolean flag = true;\\
        Boolean flag3 = true;\\
        void timeDec()\\
        \{
            int k = 20, j = 1, m = 0;\\
            
            for (int i = 0; i < 121; i++)\\
            \{
                System.Threading.Thread.Sleep(1000);\\

                this.Dispatcher.BeginInvoke(() =>\\
                \{
                    if (flag)\\
                    \{
                        if (flag3)\\
                        \{
                            string s1;\\
                            s1 = textBlock7.Text;\\
                            int num = int.Parse(s1);\\
                            num--;\\
                            s1 = num.ToString();\\
                            textBlock7.Text = s1;\\
                       
                            if (i == 120)\\
                            \{\\
                                flag = false;\\
                            \}\\
                        \}
                        else\\
                        \{
                            
                        \}
                    \}
                    else
                    \{
                       // NavigationService.Navigate(new Uri("/Page36.xaml", UriKind.Relative));\\
                    \}
                \});
            \}
            //    doStuff3();\\
        \}
        private void image1_ImageFailed(object sender, ExceptionRoutedEventArgs e)\\
        \{

        \}
        private void button1_Click(object sender, RoutedEventArgs e)\\
        \{
            NavigationService.Navigate(new Uri("/Page17.xaml", UriKind.Relative));\\
        \}
        int i = 3, m = 2;\\
        Boolean flag1 = true;\\
        Boolean flag2 = true;\\
        private void button3_Click(object sender, RoutedEventArgs e)\\
        \{
            string s1;\\
            s1 = textBox1.Text;\\
            if (s1 == "moral support")\\
            \{
                flag3 = false;\\
                button1.Visibility = Visibility;\\
                textBlock10.Text = "E";\\
                textBlock11.Text = "C";\\
                textBlock2.Visibility = Visibility.Collapsed;\\
                textBlock3.Visibility = Visibility.Collapsed;\\
                textBlock5.Text = "You got it Right. 2 of 6 letters are unlocked...Press Continue";\\
            \}
            else
            \{
                textBlock5.Text = "Sorry You are wrong !!!";\\
                if (flag1 == false)\\
                \{

                    
                \}
                textBox1.Text = "";\\
                if (m != 0)\\
                \{

                    if (m == 2)\\
                    \{
                        textBlock5.Text = "Sorry You are wrong !!!";\\
                        textBlock3.Text = "2"; m--;\\
                    \}
                    else if (m == 1)\\
                    \{
                        textBlock5.Text = "Sorry You are wrong again !!!";\\
                        textBlock3.Text = "1"; m--;\\
                    \}
                \}
                else
                \{
                    if (flag2)\\
                    \{
                        image3.Visibility = Visibility.Collapsed;\\
                        textBlock5.Text = "You lost a life !!! ";\\
                        textBlock3.Text = "3";\\
                        m = 2;\\
                        flag2 = false;\\
                    \}
                    else
                    \{
                        NavigationService.Navigate(new Uri("/Page35.xaml", UriKind.Relative));\\
                    \}
                \}
            \}
        \}
    \}
\}







\section{Prototype : 3}
\hspace{1cm}In prototype 3 implementation of timing constraints, bonus life option and climax was been carried.
\subsection{Climax}
\ttfamily \hspace{1cm}
Code skeleton:\\

namespace MD
\{
    public partial class Page33 : PhoneApplicationPage
    \{
        public Page33()
        \{
            InitializeComponent();\\
            System.Threading.Thread startupThread =\\
                          new System.Threading.Thread(new System.Threading.ThreadStart(pDelay));\\
            startupThread.Start();
        \}\\
        BitmapImage im1 = new BitmapImage(new Uri("/MD;component/Images/d.png", UriKind.Relative));\\
        BitmapImage im2 = new BitmapImage(new Uri("/MD;component/Images/dhj.png", UriKind.Relative));\\
        BitmapImage im3 = new BitmapImage(new Uri("/MD;component/Images/asa.png", UriKind.Relative));\\
        BitmapImage im4 = new BitmapImage(new Uri("/MD;component/Images/asb.png", UriKind.Relative));\\
        void pDelay()
        \{
            int k = 0, j = 1, m = 0,k1=460,k2=300,k3=400;\\
            Boolean flag = true;\\
            Boolean flag1 = true;\\
            for (int i = 0; i < 500; i++)
            \{
                System.Threading.Thread.Sleep(150);

                this.Dispatcher.BeginInvoke(() =>
                \{
                    if (flag)
                    \{
                        if (flag1)
                        \{
                            if (j == 0)
                            \{
                                image3.Source = im3;\\
                                image2.Source = im1;\\
                                image2.Margin = new Thickness(k, 222, 0, 0); k = k + 5;\\
                                image3.Margin = new Thickness(k1, 170, 0, 0); k1 = k1 - 5;\\
                                j = 1;
                                if (k > 240 && k1 < 255)
                                    flag1 = false;

                            \}
                            else
                            \{
                                image3.Source = im4;\\
                                image2.Source = im2;\\
                                image3.Margin = new Thickness(k1, 170, 0, 0); k1 = k1 - 10;\\
                                image2.Margin = new Thickness(k, 210, 0, 0); k = k + 10;\\
                                j = 0;
                                if (k > 240 && k1 < 260)
                                    flag1 = false;
                            \}
                        \}
                        else
                        \{
                            image4.Margin = new Thickness(111,k2, 0, 0); k2 = k2 - 5;\\
                            image5.Margin = new Thickness(27, k2, 0, 0); k2 = k2 - 5;\\
                            if (k3 > 270)
                            \{
                                image6.Margin = new Thickness(277, k3, 0, 0); k3 = k3 - 5;
                            \}
                            else
                                flag = false;

                        \}
                    \}
                    else
                    \{
                        NavigationService.Navigate(new Uri("/Page34.xaml", UriKind.Relative));
                    \}
                \});
            \}
            //    doStuff3();
        \}
    \}
\}

\rmfamily
